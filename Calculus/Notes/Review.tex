%% LyX 2.1.1 created this file.  For more info, see http://www.lyx.org/.
%% Do not edit unless you really know what you are doing.
\documentclass[english]{article}
\usepackage[T1]{fontenc}
\usepackage[utf8]{luainputenc}
\usepackage{amsmath}
\usepackage{esint}
\usepackage{babel}
\begin{document}

\title{Calculus 2}


\author{Professor Glaser}

\maketitle

\part{Review}


\section{Basic Antiderivatives}

Below are some antiderivatives you should be able to list at the drop
of a hat. These are fundamental to any serious problem solving.


\subsection{Exponents}

\[
\int x^{n}dx=\frac{x^{n+1}}{x+1}+C\, n\neq-1\qquad\int\frac{1}{x}dx=\ln|x|+C
\]


\[
\int e^{x}dx=e^{x}+C\qquad\int a^{x}dx=\frac{a^{x}}{\ln a}+C\,\left(a>0,a\neq1\right)
\]



\subsection{Trig Functions}

\[
\int\cos xdx=\sin x+C\qquad\int\sin xdx=-\cos x+C
\]


\[
\int\sec^{2}xdx=\tan x+C\qquad\int\csc^{2}xdx=-\cot x+C
\]


\[
\int\sec x\tan xdx=\sec x+C\qquad\int\csc x\cot xdx=-\csc x+C
\]


\[
\int\tan xdx=\ln|\sec x|+C\quad\left(=\ln|\cos x|+C\right)\qquad\int\cot xdx=\ln|\sin x|+C
\]


\[
\int\sec xdx=\ln|\sec x+\tan x|+C\qquad\int\csc xdx=-\ln|\csc x+\cot x|+C
\]



\subsection{Fractions}

\[
\int\frac{dx}{\sqrt{1-x^{2}}}=\sin^{-1}x+C
\]


\[
\int\frac{dx}{1+x^{2}}=\tan^{-1}x+C
\]


\[
\int\frac{dx}{a^{2}+x^{2}}=\frac{1}{a}\tan^{-1}\frac{x}{a}+C
\]


\[
\int\frac{1}{2\sqrt{x}}dx=\sqrt{x}+C\qquad\int\frac{dx}{x^{2}}=-\frac{1}{x}+C
\]



\section{U-Substitution}

Again, you should know this stuff. Like, you should be able to do
all of this without thinking. These problems are designed to be simple.


\subsection{Basic U-Sub}

\[
\int2x\left(x^{2}+10\right)^{98}dx
\]
We assign $u$ to the pain in the ass, so $u=x^{2}+10$ and $du=2xdx$.
Rewrite the equation so it's blantantly obvious and solve:

\[
\int2xdx\left(x^{2}+10\right)^{98}=\int u^{98}du=\frac{u^{99}}{99}+C
\]
Then you substitute in the original like so:

\[
\boxed{\frac{\left(x^{2}+10\right)^{99}}{99}+C}
\]
Not so bad, right?


\subsection{A bit harder}

All right, what if the answer isn't glaringly obvious?

\[
\int2x\left(x^{2}+10\right)^{98}dx
\]


This is also really easy. $u=x^{2}+10$ and $du=2xdx$, so just divide
until $dx$ fits in nicely. Since $\frac{1}{2}du=xdx$, we can do
the exct same thing, but with a new constant.

\[
\int xdx\left(x^{2}+10\right)^{98}=\int\frac{1}{2}u^{98}du=\frac{1}{2}*\frac{u^{99}}{99}+C
\]


Substitute again:

\[
\boxed{\frac{\left(x^{2}+10\right)^{99}}{198}+C}
\]


Again, not hard at all.


\subsection{Bounded Integrals}

What is there are bounds on the integral? This isn't hard, just more
tedious.

\[
\int_{0}^{1}2x\left(x^{2}+2\right)^{5}dx
\]


Do the same thing with $u$:

\[
u=x^{2}+2\quad du=2xdx
\]


\[
\int_{?}^{?}u^{5}du
\]


Great. Now that the equation is in terms of $u$, we have to recalculate
the bounds. Since we already know the equation for $u$, shoudln't
be that bad. Plug in 0 and 1 in this case and solve it out:

\[
\int_{2}^{3}u^{5}du=\boxed{\frac{3^{6}}{6}-\frac{2^{6}}{6}}
\]


Boom. Done.


\subsection{Fractions}

\[
\int\frac{2xdx}{x^{2}+34}
\]


If you set $u=x^{2}+34$ and $du=2xdx$, this problem is actually
stupidly simple. But you have to remember some log rules.

\[
\int\frac{du}{u}=\ln|u|+C=\boxed{\ln|x^{2}+34|+C}
\]


...Let's keep going.


\subsection{$e$ and Friends}

\[
\int xe^{x^{2}}dx
\]


Remember, $u$ handles the pain in the ass part. $u=x^{2}$and $\frac{1}{2}du=xdx$.

\[
\frac{1}{2}\int e^{u}du=\frac{1}{2}e^{u}+C=\boxed{\frac{1}{2}e^{x^{2}}+C}
\]



\subsection{Trig}

I'm too lazy to write commentary for this:

\[
\int\frac{\cos x}{\sqrt{\sin x}}
\]


\[
u=\sin x\quad u=\cos xdx
\]


\[
\int\frac{du}{\sqrt{u}}=\int u^{-\frac{1}{2}}du=2u^{\frac{1}{2}}=\boxed{2\sqrt{\sin x}+C}
\]



\section{Final Words}

I swear, Alex, if you don't remember any part of this after learning
it 4 times in your life, I want you to call Miguel and Carrie and
Alexandra so they can yell at you. A lot.
\end{document}
