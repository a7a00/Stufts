%% LyX 2.0.6 created this file.  For more info, see http://www.lyx.org/.
%% Do not edit unless you really know what you are doing.
\documentclass[english]{article}
\usepackage[T1]{fontenc}
\usepackage[latin9]{inputenc}
\usepackage{geometry}
\geometry{verbose,lmargin=2cm,rmargin=2cm}
\usepackage{esint}
\usepackage{babel}
\begin{document}

\title{Homework 18: The Integral Test and P-Series}


\author{Alexander Gould, Section 3}

\maketitle
2. Is it true as the terms of a positive series approach 0 that the
series converges? Give an example.

Yes. As the sequence approaches 0, there's less and less to add, and
the function levels off. $\left(0.6+0.06+0.006+\ldots+0.\overline{0}6=\frac{2}{3}\right)$

3. Can the integral Test be used to determine is a series diverges?

Yep! If the integral diverges, so does the series.

4. For what values of $p$ does the series $\sum_{k=1}^{\infty}\frac{1}{k^{p}}$
converge? For what values does it diverge?

It converges when $p$ is positive. When $p$ is negative, the thing
osccilates and diverges.

23. Use the Integral Test to test the convergence of $\sum_{k=2}^{\infty}\frac{1}{k\ln k}$.

$\int_{2}^{\infty}\frac{1}{x\ln x}dx$ doesn't converge to anything,
which means the series diverges.

25. Use the Integral Test to test the convergence of $\sum_{k=1}^{\infty}ke^{-2k^{3}}$.

The integral of the summed term will eventually converge to something
as the base is overpowered by the exponent, so this series converges.

28. Use the Integral Test to test the convergence of $\sum_{k=2}^{\infty}\frac{1}{k\left(\ln k\right)^{2}}$.

This integral of the summed term gradually gets bigger and doesn't
approach anything, so this series diverges.

30.

Use the Integral Test to test the convergence of $\sum_{k=2}^{\infty}\frac{1}{k\ln k\ln\left(\ln k\right)}$.

This integral of the summed term also gets bigger and doesn't approach
anything, so this series diverges as well.

31. Determine the convergence or divergence of $\sum_{k=1}^{\infty}\frac{1}{k^{10}}$.

The exponent is bigger than 1. The series converges.

33. Determine the convergence or divergence of $\sum_{k=3}^{\infty}\frac{1}{\left(k-2\right)^{4}}$.

Doesn't matter what's in the denominator, if the exponent is bigger
than 1, we converge.

43A. True or false? If a summed series starting at 1 converges, that
same series starting at 10 converges.

True. Convergence has to do with the end of a sequence, not the beginning.

43B. True or false? If a summed series starting at 1 diverges, that
same series starting at 10 diverges.

True. Dinvergence also has to do with the end of a sequence, not the
beginning.

43C. If the sum of $a_{n}$ converges, will the sum of $a_{n}+.000001$?

No. As you keep adding this new constant, the sequence gradually converges
to a linear function, not a constant.

43D. If the sum of $p^{k}$ diverges, will the sum of $\left(a_{n}+.000001\right)^{k}$?

Yes. The constant won't affect the behavior of the exponent.

43E. If the sum of $p^{k}$ diverges, will the sum of $p^{k+.000001}$?

No. Eventually, the exponent will cross 1, and the series will converge.
\end{document}
