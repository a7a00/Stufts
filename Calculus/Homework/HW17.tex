%% LyX 2.0.6 created this file.  For more info, see http://www.lyx.org/.
%% Do not edit unless you really know what you are doing.
\documentclass[english]{article}
\usepackage[T1]{fontenc}
\usepackage[latin9]{inputenc}
\usepackage{babel}
\begin{document}

\title{Homework 17: Infinite Series}


\author{Alexander Gould, Section 3}

\maketitle
47. Find the formula for the $n$th term of $\sum_{k=-1}^{\infty}\left(\frac{1}{k+1}-\frac{1}{k+2}\right)$.
Then find its limit.

The $n$th term is $1-\frac{1}{n+2}$. As $x\rightarrow\infty$, we
approach 1.

51. Find the formula for the $n$th term of $\sum_{k=-1}^{\infty}\left(\ln\frac{k+1}{k}\right)$.
Then find its limit.

The $n$th term is $\ln((n+1)!)-\ln(n!)$, which means that the series
dierges.

59A. Is $\sum_{i=1}^{\infty}\left(\frac{\pi}{e}\right)^{-k}$ a convergent
geometric series? Explain why or why not.

We can tell that the values of the summed term approach 0, so we can
conclude that the sum approaches a constant.

59B. If the sum of a series in terms of $k$ starts at $k=12$ and
converges, will it still converge if the sum starts at $k=1$?

Yes. Convergence has to do with the end of the series, not the beginning.

59C. If $\sum_{i=1}^{\infty}a^{k}$ converges, will $\sum_{i=1}^{\infty}b^{k}$
converge if $\left|a\right|<\left|b\right|$?

Yes. Bases don't matter, exponents do.

9.
\[
\sum_{k=0}^{\infty}\left[3\left(\frac{2}{5}\right)^{k}-2\left(\frac{5}{7}\right)^{k}\right]
\]


The limit of the summed term as $k\rightarrow\infty$ is 0 thanks
to the continuity of $k$ at infinity making the term go to $\infty-\infty=0$.
\end{document}
