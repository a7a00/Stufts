%% LyX 2.1.1 created this file.  For more info, see http://www.lyx.org/.
%% Do not edit unless you really know what you are doing.
\documentclass[english]{article}
\usepackage[T1]{fontenc}
\usepackage[utf8]{luainputenc}
\usepackage{url}

\makeatletter

%%%%%%%%%%%%%%%%%%%%%%%%%%%%%% LyX specific LaTeX commands.
%% Because html converters don't know tabularnewline
\providecommand{\tabularnewline}{\\}

\makeatother

\usepackage{babel}
\begin{document}

\title{Introduction to Data Structures}


\author{cgregg@cs.tufts.edu}

\maketitle

\part{Course Goals}

This course aims to teach you to program efficiently. It's not enough
to just write code. You need to refactor that stuff so it doesn't
destroy the CPU. We're going to analyze the costs associated with
coding choices and solve real-world coding problems. Hopefully, we'll
have some fun as well.


\part{4 Lists}

We want to study data structures because they teach us to manage a
ton of data efficiently in a short time frame. So:
\begin{itemize}
\item Let's create 4 ``List-Like'' containers for our data.
\item Let's add 100,000 integers to each container (the even integers between
0 and 200,000)
\item We search for the integers from 0-50,000
\item We erase the integers from 0-20,000
\end{itemize}
With a 2.4GHz processor:

\begin{tabular}{|c|c|}
\hline 
Structure & Overall (s)\tabularnewline
\hline 
\hline 
Array & 9.80517\tabularnewline
\hline 
Linked List & 23.38263\tabularnewline
\hline 
Hash Table & 0.01826\tabularnewline
\hline 
Binary Tree & 0.06077\tabularnewline
\hline 
Sorted Array & 4.87678\tabularnewline
\hline 
\end{tabular}

Data structures are no joke, obviously. But what's interesting is
when we break down the structures, different parts have different
problems. Inserting is much faster on an arrray than a sorted array,
while searching is the other way around. Why are there these discrepancies?
Well, the bottom line is that some structures contain more information
by design. And manipulating structures \textit{always} takes time.
The computer hardware knows only a giant array. Our job with C++ is
to change the way this data is laid out.

How can structures have information contained within them? Well, look
at a sorted array. Because we know that every number to the left of
the index is lower, every number to the right is higher. Unfortunatley,
this is also a tradeoff. Inserting numbers sucks in sorted arrays
because you have to run some crappy insertion sort algorithm to put
anything in the list.


\part{The Wait List and YouTube}

Everyone on the waitlist is enrolled. Done. Also all classes will
be uploaded to YouTube.


\part{Staff}


\section*{Professors}
\begin{itemize}
\item Dr. Gregg - Grew up in Upstate NY, studied EE ant Johns Hopkins, and
was recruited into the navy where he became a cryptologist. During
that time he went to San Diego, the Middle East, Thailand, Malaysia,
you get the idea. He was then posted to Australia where he went surfing
often, and decided to become a teacher, getting his Masters in Education
at Harvard. He then taught at Brookline HS down the street from Tufts,
taught at UC Santa Cruz, got his PhD in computer engineeering from
UVA, and was sent to Africa by the Navy Reserves (He's since stayed
in the Navy Reserves, and has to go to Seattle once a month.) He was
then grabbed from Africa by Tufts.
\item Bruce Molay - A cool guy who brings popcorn into lab. He wasn't here
to tell us about himself.
\end{itemize}

\section*{TA's}
\begin{itemize}
\item Tomoki Shibata
\item Hugo Akita
\item Tons of undergrads.
\end{itemize}

\part{Resources}


\section*{The Magic of the Internet}
\begin{itemize}
\item Course Website - \url{cs.tufts.edu/comp/15}
\item Piazza - Private StackOverflow for this class - \url{piazza.com/tufts/fall2014/comp15/home}
\item Some other stuff I couldn't write down because I was busy tripping
over people.
\end{itemize}

\section*{Textbooks}

Data Structures and Algorithm Analysis, 4th Edition, Mark A Weiss

\url{people.cs.?t.edu/~shaffer/Book/c++3elatest.pdf}


\part{Labs}

Lab signups will begin at 6PM \textit{tonight}. First come first served
on the class website. Labs are mandatory. 22 people per lab. You will
be given makefiles to submit the assignment.


\part{Eclipse}

Eclipse is recommended, but if you want to use the filthy piece of
trash known as emacs or the glorious paragon of simplicity known as
vim, that's great, too. Cite your fellow students. Code you submit
is owned by Tufts.


\part{Assignments}


\section*{Assignment 0}

Learn to use Eclipse and SSH into the homework server from your machine.
Make sure you use -X parameter. when you SSH. Get your CS username
so you can SSH into the machine.


\section*{Assignment 1}

Due 9/12/14. This assignment is on Dynaamic Arrays. For this assignment,
you'll be given a ton of boilerplate code, so you only need to write
13 functions.
\end{document}
