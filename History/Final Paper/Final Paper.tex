%% LyX 2.0.6 created this file.  For more info, see http://www.lyx.org/.
%% Do not edit unless you really know what you are doing.
\documentclass[12pt,english]{article}
\usepackage{mathptmx}
\renewcommand{\familydefault}{\rmdefault}
\usepackage[T1]{fontenc}
\usepackage{CJK}
\usepackage[letterpaper]{geometry}
\geometry{verbose,tmargin=1.5in,bmargin=1.5in,lmargin=1in,rmargin=1in}

\makeatletter
%%%%%%%%%%%%%%%%%%%%%%%%%%%%%% User specified LaTeX commands.
%\usepackage[utf8]{inputenc}
%This damn thing is impossible!
%\usepackage{CJK}
%\usepackage[adobefonts]{ctex}
%OKAY, LATEX. HOW ABOUT YOU GO FUCK YOURSELF?
%SERIOUSLY. GO FUCK YOURSELF. YOU'RE A TERRIBLE PROGRAM.
%FUCK YOU.

\makeatother

\usepackage{babel}
\begin{document}
\begin{CJK}{EUC-JP}{}%

\title{Humor in the Floating World: The Evolution of Humorous Manga in Japan}
\end{CJK}


\author{Alexander Gould, HIST-42}

\maketitle
\pagebreak{}

If you ask an average Westerner what Japan is known for today, chances
are good that you will get responses that highlight Japan's history
as an industrial and technological power, such as computers, robotics,
cars and Japan's ubiquitous salarymen. You might also get responses
that emphasize Japan's awareness of its own history and, such as \textit{samurai},
\textit{ninja}, swordsmanship, and traditional Japanese dishes such
as \textit{sushi} or \textit{mochi}. But the first answer you may
get will often highlight Japan's creative industries. Beginning with
Japan's economic boom in the 1980's, Japanese cultural exports have
captivated the minds-and wallets-of the world, from the world's video
game giants like Nintendo, Sony, Capcom and Namco to Japan's instantly
recognizable \textit{anime}, and its counterpart, the bound and printed
comics known as \textit{manga}. Japan's economic recovery of the 2010's,
combined with a growing interest in comics in the West and instant
distribution channels through the Internet, has made Japan's \textit{manga}
insustry boom in size, commanding 420 billion yen in Japan alone,\cite{MangaBigInJapan}
over \$250 million in Europe\cite{MangaBigInEurope}, and \$175 million
in the United States.


\part*{Origins of Manga}


\part*{Evolution of Manga}


\part*{The Edo Period}

This is where humor begins to come into its own


\section*{Ukiyo-E}


\section*{Shunga}


\section*{Kamishibai}


\section*{Kibyoshi}

\pagebreak{}

\bibliographystyle{plain}
\bibliography{Bibliography}

\end{document}
