%% LyX 2.1.1 created this file.  For more info, see http://www.lyx.org/.
%% Do not edit unless you really know what you are doing.
\documentclass[english]{article}
\usepackage[T1]{fontenc}

\makeatletter
%%%%%%%%%%%%%%%%%%%%%%%%%%%%%% User specified LaTeX commands.
\usepackage{CJKutf8}

\makeatother

\usepackage{babel}
\begin{document}
\begin{CJK}{UTF8}{}%

\title{Japan from Prehistory to 1868 (万歳!)}


\author{Professor Leupp}

\maketitle

\part{Introduction/Logistics}


\section{Syllabus}

The full copy should be available on Trunk.
\begin{itemize}
\item Lectures - 0\%
\item Discussion Sessions - 5\%
\item Midterm - 30\%
\item Take-home final - 30\%
\item Research Paper - 30\%
\item Research Presentations - 5\%
\end{itemize}
Lectures will often be straightforward and chronological, but some
will be thematic. The final exam does not require footnotes, but of
course the research paper does. Pick whatever you want, but make sure
the professor approves it first.


\section{Basic Stuff}
\begin{itemize}
\item BC and AD and BCE and CE are obviosuly the same.
\item Macrons aren't required.
\item I might be using some kanji. (大名、東京)
\item Surnames come first.
\item Romanization of Japanese is very standard, just watch out for ん, it
could be ``n'' or ``m''.
\item I'm not going to use hangeul or Chinese characters.
\end{itemize}

\section{Avoiding Bias in Japanese History}

Japanese history itself has a history. We see a movement from glorifying
it as part of the Orient (like, from the 1200's on, people thought
it was a fantasyland), to comparing it favorably to England after
the Meiji Restoration. There's a lot of history that glorifies it
militarily after the Russo-Japanese war, and demonizes it in the 1930's
and then as a prime example of modernization, especially in the 1980's.
There hae been criticisms of this, though. Reischauer's ``incipient
capitalism'' theory says that we need to use Japan as a model for
underdeveloped Asian nations, but this drive is really driven by capitalism.
\clearpage\end{CJK}
\end{document}
