%% LyX 2.1.1 created this file.  For more info, see http://www.lyx.org/.
%% Do not edit unless you really know what you are doing.
\documentclass[english]{article}
\usepackage[T1]{fontenc}
\usepackage[utf8]{luainputenc}
\usepackage{babel}
\begin{document}

\title{Japanese History to 1868}

\maketitle

\part{Geography}

Japan is an archipelago off the coast of northeast Asia, bounded by
the Sea of Japan, the Sea of Okhotsk, and the East China Sea. While
it is similar to Britain in that it's an island with a largely derivative
culture, its separation from its main continent is much worse, with
much longer distances and typhoons that thwarted the Mongols twice.
Japan had no fear of any invasion like England did. Japan has only
been successfully invaded once, in 1945. As a result, the outside
is seen as a place of good things, looms, silk, ceramics, literacy,
and of course after the Meiji restoration, ビール! It was also possible
to regulate contact. There were periods of extensive contact, followed
by periods of relative withdrawl. From 600-850, there's a period of
extensive borrowing from Choseon Korea and Tang China. Japan then
withdrew to create its own culture, which is where we get a lot of
stereotypical ``Japanese Culture''. From 1400-1640, Japan borrowed
heavily from China and Europe as ports opened for trade. Tokugawa
withdrew it again, and Matthew Perry forced it open. Japan is larger
than Germany, but smaller than France or Montana. It's vastly mountainous
and densley populated. (For now...) It's a country of volcanic activity,
and is part of the ``ring of fire'' of seismic activity as it separated
from Asia. We're due for a massive eruption from Fuji soon. Earthquakes
are common in Japan, and are a common and deadly occurence, as well
as their cousins, tsunamis. These can cause fires, collapses, floods,
and more recently, nuclear accidents. Stone is rare, and wood is used
to protect against earthquakes. Unfortunatley, this has historically
caused enormous fires, far deadlier than any in the US. Japan is made
of 4 mountainous Home Islands, producing many cultures and dialects.
Compare England for example.


\section{本州}

The main Home Island, it's about the size of England. The Kanto reigon,
built around the Kanto plain, produces lots of food. Kansai is a Western
area, open to new ideas, more trade, and is close to the Capital,
The Nobi plain surrounds Nagoya. Sado Island is nearby, home to Japan's
precious metal mining and a place of exile for political outcasts.
It's also home to Japan's largest lake.


\section{球種(?)}

Originally built of 9 kingdoms, this is the main trading point of
Japan.


\section{四国}

Separated by a beautiful inland sea with many islands and whirlpools,
this island is relatively unimportant historically. Awaji island is
connected to it by bridge and is a popular vacation spot.


\section{北海度}

Cold and isolated, this island is south of the Kuril islands, contested
between Japan and Russia.


\section{Japanese Territories}

Tsushima Island lies between Japan and Korea and is a point of tension
between the nation throughout history. The Ryukan island has since
been occupied by Japan, despite a cultural distinction. And of course,
the Senkaku Islands, site of an ongoing spat between China and Japan.


\part{Domestic Products}

Japan has an abundance of water, streams, oceans, paddies, and of
course 温泉, giving rise to a culture obsessed with bathing. However,
there are very few navigable rivers. Japan is rice country, and the
bulk of the population has been devoted to that activity, resulting
in a very high population. However, agriculture can't take up a lot
of Japan's mountainous terrain, which concentrates the population
in coastal cities. Japan is very densley populated, resulting in very
small housing spaces, many convinience stores, small cars, and heavy
reliance on mass transit. Also, the Japanese have adorable little
motorbikes. This high population density makes rice in Japan very
expensive, and it's an absolute staple. It's shorter and stickier
than its Chinese cousin, and ostensibly tastes better. (According
to Abe, at least). Sake, made from rice is incedibly important, as
are mochi and rice crackers. The cultivation and harvesting of rice
is a year long intensive process, mostly undertaken by women. There
would then be festivals of the harvest, with heavy emphasis on teamwork,
which influences Japanese group mentality today. Secondary crops include
millet, wheat, buckweed, sesame, rapeseed, and of course soy, star
of miso, tofu, natto, soy sauce, etc. Azuki is common for dessert,
and bamboo is grown as a snack and construction material. Daikon,
burdock, gobo, wasabi, eggplant, persimmons, plums, oranges, melons,
peach, ginger, konjac, and tea are all important Japanese products.
Tea is all important throughout history. Hemp was grown for cloth,
and used for medicine, but never smoked as in the west. Flax, cotton
and mulberry plants produced most of Japan's cloth, and indigo, paper
and laquer produced her artistic items. Fishing and whaling (stop
giggling, Alex) are common, as is the kollection of konbu and wakame
seasweed. Chickens and ducks were raised, but pigs wew not domesticated.
Boar was hunted and served as a delicacy in cities. Oxen and cattle
were present, but rare, and horses were used in the military only.
Buddhism also prohibited the killing of animals for meat. Human labor,
not animal, drives much of Japanese history.

Food products were taxed, by both the emperor and various daimyo.
Rice was measured in units called koku, and about 60\% of it was turned
over. 1 koku, or 330 pounds, was a year's worth of food. You would
need about 2 koku of income to survive, and peasants were encouraged
to make do with substitute grains.

\epigraph{`The proper way to rule is to make sure your peasants neither live,
nor die.''}{Tokugawa Iyeasu}

Lords and peaseants were often sworn enemies over taxes, however taxes
were rarely raised out of fear of an uprising. Peasants complain throughout
history that they and their lords are musually antagonistic
\end{document}
