%% LyX 2.0.6 created this file.  For more info, see http://www.lyx.org/.
%% Do not edit unless you really know what you are doing.
\documentclass[english]{article}
\usepackage[T1]{fontenc}
\usepackage[latin9]{inputenc}
\usepackage{textcomp}
\usepackage{amstext}

\makeatletter

%%%%%%%%%%%%%%%%%%%%%%%%%%%%%% LyX specific LaTeX commands.
\newcommand{\lyxmathsym}[1]{\ifmmode\begingroup\def\b@ld{bold}
  \text{\ifx\math@version\b@ld\bfseries\fi#1}\endgroup\else#1\fi}


\makeatother

\usepackage{babel}
\begin{document}

\title{�1.5, \#28}


\author{Alexander Gould}

\maketitle
A is true. All real numbers, when squared, yield another real number.
Since there's no upper limit on real numbers, there will always be
some real number $y$ that satisfies the condition, no mater the size
of $x$.

B is true. For all $x$, there exists some $y$ that is $\sqrt{x}$,
which satisfies the condition $x=y^{2}$.

C is true. There does exist some real number $x$ that will yield
0 when multiplied with all real numbers. It's 0.

D isn't true. In fact, the Commutative Property of Addition is written
$\forall x\forall y\left(x+y=y+x\right)$, and that's an established
truth. D is literally a straight negation of that, which makes it
false.

E is true. This satement just says that for any real number that isn't
0, there's some real number that multiplies with it to make 1. Since
a number and its reciprocal always multiply to make 1, and every nmber
has a reciprocal, this is true.

F is false. This statement says that there's some real number that
multiplies to 1 with all nonzero real numbers. This is impossible.
Whenever only one term in multiplication changes, the product changes,
unless the other term is 0, which it isn't.

G is true. All numbers have a number that, when added yield 1. You
can find this number with the equation $y=1-x$.

H is false. There are no values of $x$ and $y$ that fulfill the
2 eauations $x+2y=2$ and $2x+4y=5$. If you graph them, they're parallel
lines.

I is false. The 2 equations $x+y=2$ and $2x\lyxmathsym{\textminus}y=1$
are only true when $x$ and $y$ are both 1. There is no other value
of $x$ where both are true, making this statement false.

J is true. No matter which $x$ and which $y$ you pick, there will
always be an average between the two of them.
\end{document}
