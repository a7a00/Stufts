%% LyX 2.0.6 created this file.  For more info, see http://www.lyx.org/.
%% Do not edit unless you really know what you are doing.
\documentclass[english]{article}
\usepackage[T1]{fontenc}
\usepackage[latin9]{inputenc}
\usepackage{textcomp}
\usepackage{babel}
\begin{document}

\title{Star Problems}


\author{Alexander Gould}

\maketitle

\section*{� 1.8, \#6}

Prove using the notion, without loss of generality, that $5x+5y$
is an odd integer when x and y are integers of opposite parity.

For the sake of argument, let $x$ be odd and $y$ be even. This means
that the equation can be represented as $5\left(2a+1\right)+5\left(2b\right)$,
for 2 integers $a$ and $b$. We can rewrite this as $10a+10b+5$,
or $2\left(5a+5b\right)+5$. Since $5a+5b$ is an integer, this equation
can be written as $2k+5$, or $2k+1+4$ for some integer $k$. Since
4 is even, we can further write this as $2k+2j+1$, where $j=2$.
Simplify this to $2\left(k+j\right)+1$, and since $k+j$ is an integer,
we can further simplify it to $2h+1$, the definition of an odd integer.


\section*{� 1.8, \#20}

Prove that given a real number $x$, there exist unique numbers $n$
and $a$ such that $x=n+a$, $n$ is an integer, and $0\leq a<1$.

If we choose an integer $b$ to be the greatest integer so that $b\leq x$
and let $a=x-b$, we know through the priciples of subtraction that
$a\geq0$. If $e\geq1$, (which would violate uniqueness), then we
can use subtraction to show $(a-1)+1=x-n$, and $n+1=x-(a-1)$. Problem
is that this implies $n+1\leq x$, which we know isn't true. Therefore,
$a$ has to be somewhere between 0 and 1. We know they're unique by
assuming 2 integers, $c$ and $d$ add up to $x$. We now know that
$c+d=n+a$, and therefore that $c-n=a-d$. If $d\geq0$ and $n<1$,
then $\left|a-d\right|<1$. Since $a$ and $d$ are both integers,
that means they'd have to be equal. Since $c-n=a-d=0$, $c=n$ and
$a=d$, which proves $n$ and $a$ are unique.
\end{document}
