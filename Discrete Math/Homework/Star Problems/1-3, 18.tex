%% LyX 2.0.6 created this file.  For more info, see http://www.lyx.org/.
%% Do not edit unless you really know what you are doing.
\documentclass[english]{article}
\usepackage[T1]{fontenc}
\usepackage[latin9]{inputenc}
\usepackage{textcomp}
\usepackage{babel}
\begin{document}

\title{�1.7, \#18}


\author{Alexander Gould}

\maketitle
\textit{What is wrong with this argument? Let $S(x,y)$ be \textquotedblleft{}x
is shorter than y.\textquotedblright{} Given the premise $\exists sS(s,Max)$,
it follows that $S(Max,Max)$. Then by existential generalization
it follows that $\exists xS(x,x)$, so that someone is shorter than
himself.}

There are 2 main problems here:
\begin{enumerate}
\item It doesn't follow at all that Max can replace $s$. All the $\exists$
sign says is that there's \textit{somebody} who's shorter than Max.
That someone doesn't have to be Max. If instead of $\exists$, we
had $\forall$, then it would logically follow, because the statement
has to hold for everybody. But $\exists$ means that we only need
at least one person to meet the criteria, and that person doesn't
have to be Max.
\item The one person in the group who meets the criteria can't be Max at
all. An essential precondition for $S(x,y)$ being true is that the
two arguments are different, so they can be compared to each other.
If you compare an object to itself, it fails that precondition, which
means $S\left(Max,Max\right)$ has to be false.\end{enumerate}

\end{document}
