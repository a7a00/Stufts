%% LyX 2.0.6 created this file.  For more info, see http://www.lyx.org/.
%% Do not edit unless you really know what you are doing.
\documentclass[english]{article}
\usepackage[T1]{fontenc}
\usepackage[latin9]{inputenc}
\usepackage{textcomp}
\usepackage{amstext}
\usepackage{amssymb}

\makeatletter

%%%%%%%%%%%%%%%%%%%%%%%%%%%%%% LyX specific LaTeX commands.
\newcommand{\lyxmathsym}[1]{\ifmmode\begingroup\def\b@ld{bold}
  \text{\ifx\math@version\b@ld\bfseries\fi#1}\endgroup\else#1\fi}

%% Because html converters don't know tabularnewline
\providecommand{\tabularnewline}{\\}

\makeatother

\usepackage{babel}
\begin{document}

\title{Homework 3}


\author{Alexander Gould}

\maketitle
91

4. Show that the negative of an even number is even. Use a direct
proof.

\begin{tabular}{|c|c|}
\hline 
Statement & Reason\tabularnewline
\hline 
\hline 
$n$ is even. & Given\tabularnewline
\hline 
$\exists k\left(n=2k\right)$ & Definition of an even number\tabularnewline
\hline 
$-n=2*-k$ & Multiply both sides by -1.\tabularnewline
\hline 
$-n=2a$ & Assign $a=-k$.\tabularnewline
\hline 
$-n$ is even. & Definition of an even number\tabularnewline
\hline 
\end{tabular}

6. Use a direct proof to show that the product of two odd numbers
is odd.

\begin{tabular}{|c|c|}
\hline 
Statement & Reason\tabularnewline
\hline 
\hline 
$a$ and $b$ are odd. & Given\tabularnewline
\hline 
$\exists c\left(a=2c+1\right)$,$\exists d\left(b=2d+1\right)$ & Definition of an odd number\tabularnewline
\hline 
$ab=\left(2c+1\right)\left(2d+1\right)$ & Multiply $a$ and $b$\tabularnewline
\hline 
$ab=4cd+2c+2d+1$ & FOIL\tabularnewline
\hline 
$ab=2\left(2cd+c+d\right)+1$ & Distributive Property\tabularnewline
\hline 
$\left(2cd+c+d\right)\in\mathbb{Z}$ & The sums and products of integers are integers\tabularnewline
\hline 
$ab$ is odd. & Definition of an odd number\tabularnewline
\hline 
\end{tabular}

11. Prove or disprove that the product of two irrational numbers is
irrational.

We can disprove this by counter-example. $\sqrt{3}$ is irrational,
but the product of $\sqrt{3}$ and $\sqrt{3}$ is either 3 or -3,
both rational.

16. Prove that if $m$ and $n$ are integers and $mn$ is even, then
$m$ is even or $n$ is even.

Using our answer from Question 6, we know that if both $m$ and $n$
are odd, $mn$ will always be odd. Therefore, at least one of them
\textit{has} to be even.

17. Show that if $n$ is an integer and $n^{3}+5$ is odd, then $n$
is even using:

A. A proof by contraposition (If $n$ is odd, $n^{3}+5$ is even.)

\begin{tabular}{|c|c|}
\hline 
Statement & Reason\tabularnewline
\hline 
\hline 
$n$ is odd. & Given\tabularnewline
\hline 
$\exists k\left(n=2k+1\right)$ & Definition of an odd number\tabularnewline
\hline 
$n^{3}+5=2\left(4k^{3}+6k^{2}+3k+3\right)$ & FOIL\tabularnewline
\hline 
$\left(4k^{3}+6k^{2}+3k+3\right)\in\mathbb{Z}$ & The sums and products of integers are integers\tabularnewline
\hline 
$n^{3}+5$ is even. & Definition of an even number\tabularnewline
\hline 
\end{tabular}

B. A proof by contradiction

We start by assuming both $n$ and $n^{3}+5$ are odd. We can use
the reasoning from Part A up until the end of the proof, where we've
proved that $n^{3}+5$ has to be even. Since there's a contradiction,
we know that either $n$ or $n^{3}+5$ is actually even. Since $n^{3}+5$
is odd is a given, we know $n$ is even.

24. Show that at least three of any 25 days chosen must fall in the
same month of the year.

If we assume that this is false, that means at most 2 days can be
in the same month, meaning we'd need 24 days max. Since we know there
are 25 days, there is a contradiction and the statement is true.

26. Prove that if $n$ is a positive integer, then $n$ is even if
and only if $7n+4$ is even.

(Not sure how to put this one in a table.) Using our answer from Question
16, we know that we need at least one even factor to get an even product.
Since 7 is odd, $7n$ will have the same parity as $n$. We can then
use the definition of an even number to say that an even number stays
even only when another even number is added to it. (Othereise the
``+1'' would turn it into an odd number.) Likewise, an even number
added to an odd number yields an odd number. (Otherwise, the 2 ``+1'''s
would cancel each other out.) Therefore, we know that $7n+4$ will
always have the same parity as $n$. Since the 2 are equivelant, they're
dependent on each other, proving the ``if and only if'' true.

28. Prove that $m^{2}=n^{2}$ if and only if $m=n$ or $m=\lyxmathsym{\textminus}n$.

Since we know the absolute values remain the same (two identical positive
integers squared will still be equal), we just need to prove that
a negative number times a negative number is positive. You can do
this by factoring out the two impied ``-1'''s, squaring both now-identical
positive numbers, and then reapplying 1 ``-1'' on each side of the
equals sign. The numbers will now be negative, but still be equal.
Since we now know that both a positive and a negative squared will
yeild a positive, and we know that the absolute value is the same,
we know that the statement is true. If $m$ is any value other than
$n$ or $-n$, the absolute value part won't hold up. Since the condition
goes both ways, the ``if and only if'' is true.

30. Show that these three statements are equivalent, where $a$ and
$b$ are real numbers: (i) $a<b$, (ii) $\frac{a+b}{2}>a$, and (iii)
$\frac{a+b}{2}<b$.

If the statements are equivelant, we should be able to get any statement
from any other. So $a+b>2a$, $b>2a-a$, $a<b$ gets us from ii to
i. Likewise, $a+b<2b$, $a<2b-b$, $a<b$ gts us from iii to i. And
we can get from i to iii via $a+b<b+b$, $\frac{a+b}{2}<\frac{2b}{2}$,
$\frac{a+b}{2}<b$, and from i to ii via $a+a<b+a$, $\frac{2a}{2}<\frac{a+b}{2}$,
$\frac{a+b}{2}>a$.

38. Find a counterexample to the statement that every positive integer
can be written as the sum of the squares of three integers.

15. It's greater than the squares of the first 3 squares, so those
are the only ones we can use. But no combination of those gets us
15. 
\end{document}
