%% LyX 2.0.6 created this file.  For more info, see http://www.lyx.org/.
%% Do not edit unless you really know what you are doing.
\documentclass[english]{article}
\usepackage[T1]{fontenc}
\usepackage[latin9]{inputenc}
\usepackage{geometry}
\geometry{verbose,lmargin=1.4cm,rmargin=1.4cm}
\usepackage{setspace}
\doublespacing
\usepackage{babel}
\begin{document}

\title{Guilt and Shame-Based Societies in \textit{A Song of Ice and Fire}}


\author{Alexander Gould}

\maketitle
Any society worth its salt has to be ready and willing to maintain
law and order. The concept of an ordered and lawful society existed
well before there were even formal laws; early men, both in our world
and Westeros, grouped together to form tribes and clans, organizations
of mutual benefit, held together by societal rules. \cite{WOIAF}
As tribes grew into civilizations, their societies grew too large
to police themselves, and as a result, we see the development of justice
systems to enforce a strict code of behavior. Of course, a formal
justice system can't to all the work; it's purpose is to punish those
who stray too far from societal norms. It's the job of society to
police people into behaving properly in the first place, and impose
values that allow all of its members to benefit. In order to keep
people in line, societies need to use emotions that motivate people
to police their own behavior. The emotions that encourage self-policing
fall under two categories, \textit{guilt} and \textit{shame}. \cite{Greeks}
These 2 emotions are often confuced due to their similar effects,
but they're actually diametric opposites. The definitions of these
words are varied and hard to define, so it's easier to look at the
two types of societies they spawned. Every culture in the world has
had either a guilt-based society, or a shame-based society. The two
are omnipresent, as they are necessary for society to function, yet
fundamentally different. Let's take a look at the two and see why.

Guilt-based societies are centered entirely around the individual.
If an individual in a guilt based society does something wrong, they
are expected to feel \textit{guilty}. It doesn't matter what the others
or the collective think of me; if I've done something wrong in a guilt-based
society, I'm expected to feel bad about it. Consequently, I avoid
doing bad things to avoid this feeling, as a consequence of my actions.
This is a double edged sword, however; while I am expected to carry
my guilt regardless of what others think, I also reserve the right
to carry my innocence regardless of what others think. If society
turns against me, I am able to turn back and protest my innocence
because the entire system is built on individual accountability. Long
story short, a guilt-based society revolves around the idea that you
shouldn't do bad things because of a potential negative impact for
you and you alone. Guilt based societies have existed throughout history.
The most famous example would probably be Ancient Rome, which had
a system of courts,\cite{Cicero} a representative government with
emphasis on accountability,\cite{Cicero} and a culture of civic virtue,
where people were expected to be noble and upstanding citizrens of
Rome for their own individual reward.\cite{Cicero} There are other
examples of guilt-based societies, too. Mughal India featured a similar
system, where rulers and subjects were expected to hold themselves
to an individual standard of ethics, and to police themselves when
neccessary. \cite{Mughals} Europe developed the concept of individual
humanism, which placed emphasis on the individual and his public and
private actions, as early as the 1400's, and passed Enlightenment
values of self-control on to the United States in the 1700's. Guilt-based
societies are prevalent in the West, and have influenced our culture
for centuries. Our great stories feature individual heroes, fighting
against society's wrongs, and the American Dream of self-reliance
and self-policing is about a guilt based as you can get.

Shame-based societies are the exact opposite. These societies are
centered around the collective and the group. In a shame-based society,
what I think doesn't matter as much as what others think of me. As
a result, people in a shame-based society are expected to put others
before themselves. This manifests itself in multiple ways. Often individuals
subsume their own needs and wants to the needs and wants of their
relatives and communities, and shame-based societies are far more
likely to feature extremley strong family groups. Governments of shame-based
societies will often be modeled after a familial structure, as the
family is the first established community individuals see. In some
cases, such as in China, the Imperial government named positions and
relationships after familial terms. \cite{Chinese} A wrongdoer in
a shame-society who keeps his wrongdoing secret has no compunction
to feel bad at all. After all, if I've done something bad and nobody's
found out, there's been no harm to the people around me, and I shouldn't
care. Unfortunatley, if I \textit{am} found out, I have no right to
turn against society. In shame-based societies, the testimony of others
will often come before an accused's own defense and the result will
often by handed down by group consensus. An individual in a shame
society, while less obligated to feel bad about wrongdoings that aren't
found, has no right to defend themself when others find out their
wrongdoing. A shame-based society revolves around an idea that you
shouldn't do bad things because of a potential negative impact for
everyone around you and everyone you care about. Shame-based societies
have been around just as long as guilt-based ones, but are more of
a hallmark of Eastern civilization. In China, the \textit{Analects}
stress loyalty and duty to one's family and community, as well as
duty to one's government. \cite{Confucius} Japan is world-famous
for its heavy emphasis on shame and honor, and even today in modern
Japan, suicide as a result of lost societal honor is a common and
acceptable practice. \cite{Japanese} All forms of Japanese government,
from the early Yamato clan-honor cult to the Warring States Period's
samurai code to the Tokugawa Shogunate's 5-man teams, where all 5
members reported on each other, have been entirely based on shame,
where the wrongdoing's consequence is not on you, but on the people
around you. The Japanese commit seppuku because the thought of living
with the dissapointment of the group carries so many horrible consequences
that winning some respect back is worth dying. \cite{Japanese} Korea,
of course has a similar society, from the early Silla and Paekchae
kings all the way to the present. North Korea has an enormous emphasis
on the group and a national community centered around the Kim family,
while South Koreans are among the most social-media addictied people
in the world. \cite{Koreans} In either Korea, what others think of
you is of vital importance to your societal standing, be it a better
job and more prestige in the South, or a home with running water and
better rations in the North. Of course, Communist countries thrive
in and foster shame-based societies. In the USSR, people huddled together
in groups to share basic neccesities and knowledge of how to get by,
and there was a fierce loyalty in the USSR and Russia today to ones
friends and comrades. Interestingly enough, there's an example of
a western shame-based society as well, and that's England. England
has a long history of guilt based society, from the Book of Common
Law, to the Magna Carta, to the British justice system. But England
also places an enormous emphasis on national identity and fealty to
the government. The US fights for ``freedom, liberty and the pursuit
of happiness'' while Britain fights ``for King and Country'', queueing
and Brititsh politeness are common, and there's a huge loyalty to
football clubs and social clubs that isn't common elsewhere in Europe.
It seems as Britain became an empire, it developed the sense of national
identity to make something of a transition.

Before I go any further, there's something very important I need to
talk about. While these two types of society are extremley different,
neither can be objectively ciamed to be better than another. Certain
advanced and modern nations, like us, embrace the ideas of a guilt-based
society, civic virtue and self-regulation as a foundation of their
society and culture. However, other higly developed nations with high
standards of living, such as Japan and Korea, prioritize harmony,
community, and a shame-based society. Libya has had a guilt-based
society, from the time of the caliphs of North Africa, possibly even
earlier if we count Rome and Carthage. Yet Libya has an incredibly
low standard of living and has been racked by near constant war and
brutal dictatorships. Russia has had a shame-based society from the
time of Peter the Great, and the standard of living in Russia has
always lagged behinf that of Europe. What's important to note here
is that while the choice of a guilt or shame-based society can influence
how a society acts, neither is better than the other. They are simply
opposites.

So, let's get to Westeros! We'll start by looking at the noble houses.
Many of the Westerosi houses have guilt cultures, notably the Lannisters,
who have such a strong sense of individuality that they will break
family ties to get what they want, or in the case of Cersei and Jamie,
family taboos such as incest. Generally, the southern kingdoms have
more guilt-based societies. Dorne, for example has an enormous emphasis
on individual accountability, and that people should be free to do
what makes them happy, \cite{book2} While the Tyrells prize individual
actors like Margery and Olenna. I'd hazard a guess that this is probably
based on their proximity to King's Landing. Since the Targaryens founded
the city as the capital, it had been the center of Targaryen power
for hundreds of years. As the Targaryens set up political institutions,
King's Landing became the center of guilt-based power in Westeros
as the Targaryen government put into place systems that favored individual
ties and service to one's own aims. \cite{book1} Houses farther away,
like the Tullys, Greyjoys, Starks and Boltons, have a far greater
emphasis on loyalty and community. The Tullys' own motto is ``Family,
duty, honor'', while the Starks value the honor code of Lord Eddard,
who executes his own criminals and prizes honor above all else, willing
to trust the Freys at face value due to a prior relationship. Even
the Boltons' have an enormous emphasis on community, even if their
code of ethics would be considered by us as downright cruel.

Of course there are those who exist outside the Westerosi house system.
The raiders of the Iron Islands and the wildings have almost universally
developed shame societies, possibly due to the scarcity of resources
and constant warfare. Theon mentions many times the nessecity of the
Iron Islanders to stick together and perservere as a group, \cite{book1}
while the WIldings fight in clan groups, owing their allegiance to
the survival of the group. In addition, their constant adversarial
relationship to any individual authority from the Seven Kingdoms might
have caused this pattern to strengthen, similar to how Japan and Britain's
shame societies strengthened in response to perceved ``others''.
The Wildings are united, by spirit if not by group, against the enemy
on the Wall, while the Iron Islanders are united against the Westerosi
coast dwellers, who enjoy resources that the Iron Islanders lack.
Theon mentions several times that the Iron Islands do not enjoy the
same conviniences, and as a result have a different code of ethics,
as Westeros. \cite{book3} Even Stannis Baratheon begins to develop
characteristics of the ruler of a shame society, expecting loyalty
and self policing far more than Robert ever did, but only once there
was a clearly defined ``other''. Originally, Stannis was perservering
against a society that he believed had done him wrong, hallmark concept
of the guilt-based society of the Baratheon family. But once Westeros
became a monolithic ``other'', and once people like Davos Seaworth
began pledging loyalty to him, he quickly egins to expect the societal
policing that a shame based society would have. His religious split
and alliance with Melisandre, the priestess of a shame-based religion,
serves to further define Westeros with the other, and align him with
the cult of the Lord of Light, a religion woth a constant focus on
unity and purpose, with all its members acting as one. I believe this
firther pushes Stannis towards a shame-based system of rule.

In addition to the houses and outcasts that jockey for power, Westeros
has its fair share of transients. Maesters, priests, and the like
occupy a special place in Westerosi society. Ostensibly outside the
realm of politics, and with no clearly defined ``other'', we see
a guilt society emerge as most maesters and priests see themselves
as alone against the world. The very first maester we see in the books
has a sense of individuality and personal knowledge and accountability
that clashes with the honor based system of the Starks he serves.
\cite{book1} In the case of Pycelle and Varys, and even lords like
Littlefinger, not clearly in the social structure, we see huge amounts
of personal accountability and guilt, and a complete disregard of
what society thinks. Varys and Pycelle both outright say to Tyrion
that they're in this game for nothing more than to save their own
skin \cite{book2,book3} But, what's interesting to note about Littlefinger
is that as he moves up in the politics of Westeros and becomes more
powerful in the traditional sense, he starts to see himself as more
accountable to the community, and is more willing to forgive himself
for secret wrongdoings-traits that are evident of a shame society-which
he didn't do at all in the early books. COmpare his behavior and attitude
towards Ned Stark as he prepares to frame and execute him to his reaction
to the power politics surrunding the death of Joffrey. \cite{book3}
We see Littlefinger moving towards a more shame-based code of ethics
as he works his way up. Westeros is a guilt, based society, though.
Why would getting closer to the center of a guit-based society cause
Littlefinger to see himself in more of a shame-based light? The answer
is that it's the only way to keep moving up. Littlefinger thrives
by being different, by seeing advantages no one else can and taking
them. I believe what makes him so dangerous is his ability to switch
between guilt-based and shame-based ways of looking at the world.
It gives him a sense of perception that most other members of Westerosi
society lack.

As a continent defined by power change from invasion and steady and
constant dynasties that span centuries, we'd expect Westeros to have
a similar society to nations with similar histories, like Britain
and Japan. But this isn't what happens. The concept of the Targaryen
dynasty as an ``other'', with a different culture and a less than
ideal regard for Westerosi culture, indigenous reiligion, and human
life, seems to have prompted a need for reliance on oneself in some
of the houses that hoped to topple Targaryen rule. Success under these
systems gave the serious contenders for the Iron Throne lasting guilt-based
societies. Areas left alone by the Targaryens, such as the North,
not only retained their own customs and religions, but their shame-based
societies as well.

Essos has a completley different political situation. While Westeros
is culturally unified, Essos is a patchwork of different cultures,
as well as remnants of older cultures that still somewhat exist. Government
in Essos is not unified, and there is very little communication between
different parts of the continent. Essos doesn't have a small group
of outsiders like Westeros does, it has multiple small groups of ``insiders'',
and many people without a group or identity at all.

First up are the Free Cities. The history of these cities has always
been one of a struggle for freedom and independence. These seperate
city-states are born from a fight for independence from a large empire,
and as a result have about as guilt-based a culture as you can get.
There's a strong effort on commerce and cooperation, but not community
or duty; free people in these cities can come or go as they wish without
any sort of permission. The lack of community and sense of individual
needs placed first also gives us the idea that slavery is okay, as
it's a means to a commercial end. In many respects, the Free Cities
are amoral, they don't realy prohibit any activities so long as they
aren't harmed by them, and the culture of many of them, Braavos especially,
revolves around the accumulation of wealth, not any sort of prestige
or honor. The Free Cities' foreign policy reflects this; they are
largely apolitical, backing whoever they think will either pay them
money, or in the case of Braavos, get Westeros to pay its debt.

The cities that line Slaver's Bay are almost like mirror images of
the Free Cities. They have economies based on slavery and are also
largely apolitical, but they have a strong sense of identity and community.
Slavery is not justified because you don't care about the people in
your community, or value money over it, as in the Free Cities, but
rather because the slaves are not in your communtity in the first
place. Slaves are almost always considered captives, and this status
is passed down from generation to generation of slaves. The dynamic
here is extremley interesting as well; the slaves hate the Masters,
but they do not reject the idea of an Astapori or Yunkish or Meerenese
community, and a good many of them try to reintegrate into it following
Daenerys' liberation of the city. My theory is that the idea of a
Ghiscari empire lives on in these cities, and the prestige of the
community is a justification for it to place itself above all others,
almost like a hyper-shame society, and much like Old Ghis, the greatest
crime is not belonging to it.

Beyond the reach of wither of these groups of cities lie the plains
inhabited by the nomadic Dothraki, who have a strange sort of hybrid
between a guilt-based society and a shame-based one. There's an extreme
emphasis on community, but it's a community held together by antagonism
between individuals. Only by achieving individually can one inspire
the rivalry that keeps the group together, and only by this rivalry
can one hope to win notoriety, be it at a wedding to win the affection
of a dancer, or challenging a Khal over control of the horde. \cite{book1}
But how can this be? If guilt and shame-based societies are so mutually
exclusive, how can we have a society that's a hybrid of both? I'd
argue that the Dothraki have a fundamentally shame based society,
but one based on a community that's so distrustful of any outsider
that it can only survive through distrust and rivalry. The greatest
sin against the collective Dothraki society is complacency. The Dothraki
never stop, they never rest and they never let their guard down. The
attitudes towards those who break this code by failing in battle or
not providing adequate protection are consistent with the attitude
of a shame society towards its transgressors. As a result, Dothraki
shame society encourages its members to be hyper-alert, hyper-mobile,
hyer-aggressive, and unfortunatley hyper-violent against others in
order to advance and keep the community healthy.

Beyond much of Essos, near the lands of the Shadow, lie the cities
of Qarth and Asshai. I'm grouping these two together, not because
they have a similar culture, (they don't) but becase the pattern that
gave rise to their societies are almost exactly the same. These are
cities that are defined as outsiders by default. As people close to
the Shadow, the residents of these cities seem to belive the social
rules of the rest of the world don't apply to them. Case in point,
we've never met a truly ``normal'' person from either city, and
characters describe them as strange places where stranger people seem
to come from. \cite{book4,book3,book5} Qarth operates according to
a set of very complex societal rules regarding the use of trade and
magic that seem intentionally designed to be difficult for outsiders
to understand, \cite{book2} while Asshai openly flaunts the magical
arts and sees itself as a city set apart from the rest of the world.\cite{book4,book5}
As people who identify more as outsiders than residents of their cities,
guilt-based culture reigns supreme here, even if it's nearly unrecognizable
due to the residents' morality, which makes sense perfectly to them,
but leaves outsiders and the reader scratching their heads.

As a continent with a far longer history than that of Westeros, categorized
by empires rising, falling, and leaving little pieces of identity
behind, we'd expect close-knit shame societies. But, like Westeros,
the opposite of what we expect happens. The sheer magnitude of so
many societies shrinks each one down to a point of almost no consequence,
meaning that the prevailing attitude on the continent is guilt-based,
as opposed to the more shame-based natural attitudes of a more politically
static Westeros.

Westeros and Essos are extremley different continents with extremley
different societies, with Westeros being the politically constant
continent and Essos being one of near constant change. On both of
these continents, we see guilt and shame-based societies arise as
the result of pressure form outside groups, as in the case of Essian
cities, pressure from a need to adapt, as in the case of the Westerosi
houses, or plain environmental needs, as in the case of the Dothraki.
But while Westeros is characterized by a struggle between large political
blocs, and a struggle between these 2 kinds of society, Essos is characterized
by constant political upheaval, making the societies far more likely
to interact. This continual release of friction between these two
kinds of societies is, I think, part of the reason why Essos is decentralized
and nearly constantly in a state of small scale war, unlike Westeros,
which is rarely in a state of war, but when it is, it's a state of
total war. Power dynamics in the Song of Ice and Fire can be seen
as a constant struggle, between the guilt based societies and the
shame based ones. The two societies are largely incompatible in Martin's
world, and the long history of war reaching back to Ghiscari shame
society and Valyrian guilt society will, I believe continue. At least,
until one society fully wins.

\bibliographystyle{plain}
\bibliography{Bibliography}

\end{document}
