%% LyX 2.0.6 created this file.  For more info, see http://www.lyx.org/.
%% Do not edit unless you really know what you are doing.
\documentclass[english]{article}
\usepackage[T1]{fontenc}
\usepackage[latin9]{inputenc}
\usepackage{geometry}
\geometry{verbose,lmargin=2cm,rmargin=2cm}
\usepackage{babel}
\begin{document}

\title{Kingsguard/Queensguard Assignment}


\author{Alex Gould, GoT and Political Theory}
\maketitle
\begin{enumerate}
\item \textbf{Margery Tyrell} - OK, so she's better at manipulating things
behind the scenes, I get that. But I think that if I were to become
King of Westeros, I'd want someone with good PR skills and an established
family that I could trust. Not to mention she's become the inheritor
of Olenna's legacy, and given that the Tyrells are pretty much making
a naked bid for power at this point, I'd want to be on the right side
of history. As someone who's smart, ambitious and unassuming, she
would be an ideal lead member of the Kingsguard, a position that a
lot of people don't expect to have political aspirations. (Unless
you ask Ser Jamie, of course.)
\item \textbf{Brienne of Tarth} - Brienne has had it rough throughout most
of the series; both her attempts to be a proper lady and her attempts
to be a warrior have been laughed off for different reasons. Nonehteless,
she demonstrates that she has the strength of character to simply
not care. She routinely des what is right and chivalrous not for any
expectation of a reward, but simply to protect those who need help.
Were I King, I think I would strive to make myself worthy of Brienne's
protection. She's a strong moral compass and an extremley competent
fighter, and would probably be the closest thing the cynical world
of Westeros has to an ideal knight.
\item \textbf{Sandor Clegane} - Sandor actually popped into my head because
I was rewatching his and Brienne's no-holds-barred beatdown. In a
way, Sandor is Brienne's opposite. He sees the ideal society for what
it truly is with a lot more clarity than others do. He's a fantastic
judge of character and is immediatley able to see deciet and trickery,
which is naturally a trait I'd want. The main reason I'd place myself
under Sandor's protection is that he lacks a lot of what we've been
talking about in regards to the perception of power. He's relentless
and will ferociously attack whoever threatens him or the people he
protects, no matter who the threat is coming from. The ability to
see past social convention and single-mindedly set about doing what
is needed is a rare and valuable trait in Westeros, and I think this
insight would make Sandor a valuable addition.
\item \textbf{A wealthy Meereen slaver} - We're making huge assumptions
here, but I would love to have one of these people's armies of Unsullied
at my service. I see only two main obstacles to this. (We're assuming
I can get all the Unsullied over from Essos without a problem and
that Daenerys doesn't pose a threat at this time)

\begin{enumerate}
\item \textbf{Morals} - Slavery on an Ghiscari scale is relatively unknown
in Westeros, which raises an interesting question of what would happen
were there to be an army of Unsullied who answered to the crown. I
personally don't believe that this would be an enormous issue for
a very simple reason. Martin routinley implies, and sometimes states
outright, that Essos is more advanced than Westeros. The Free Cities
are in the midst of a Rennaisance-style rebirth of knowledge and are
currently booming economically and culturally from the traditions
of Old Ghis, which is a very obvious paralell to Rome. However, as
the Free Cities continue to advance, slavery only becomes more and
more prevalent, with cities like Lys operating almost entirely on
the enslavement or heavy indentiture of lots of people. Regardless
of the system, it could very easily be brought over to Westeros, which
has working conditions that are even more apalling, with decentralized
serfdom and backbreaking taxes to pay the heavy debt to the Iron Bank.
One could argue that even slavery would be preferable to a system
like this. The whole castration thing may be another issue.
\item \textbf{Logistics} - Slightly more challenging is the question-how
are you going to feed and clothe soldiers who directly answer to you
with no lords to take care of the intermediate steps? This starts
to fall into the area of tax reform and direct royal governance of
land, which would be the topic of a whole other paper. At the core
of this issue is doing it without disenfranchising the lords who currently
own the land needed to sustain this new group of soldiers, as you
never want the Kingsguard to have more work to do. I could concievably
think up a food for protection idea which might serve to centralize
Westeros in the long run, but again, other paper. What's important
is that sa system of having lards pay for their protection from the
Kingsguard would strengthen my position of royal authority, put enemies
either in debt to me or on the defensive from my armies (which we
can imagine growing much bigger and sustaining itself after the death
of the original Essians if the process of slavery to the crown caught
on in Westeros), and readies me for an eventual confrontation with
a certain Breaker of Chains, if she decides to \textit{ever} show
up.
\end{enumerate}
\item \textbf{Ser Alliser Thorne} - Sure, he's not the nicest person. Sure,
Jon hates him. But in all my searching, I'm hard pressed to find a
character with more practical experience than Ser Thorne. He knows
the ins and outs of battle almost perfectly, both from his experiences
as a warrior in Westeros, and of course in his many battles since
he's taken the black. He's a unique pragmatist and a capable commander,
who's able to keep his cool in almost any situation. His mind isn't
clouded by opinion, as are those of so many already on my list, and
he's been shown to perform well under pressure. He's also near incorruptible.
I would gladly trust someone so dead-set on his mission to protect
me.\end{enumerate}

\end{document}
