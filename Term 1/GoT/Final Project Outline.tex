%% LyX 2.0.6 created this file.  For more info, see http://www.lyx.org/.
%% Do not edit unless you really know what you are doing.
\documentclass[english]{article}
\usepackage[T1]{fontenc}
\usepackage[latin9]{inputenc}
\usepackage{geometry}
\geometry{verbose,lmargin=1.4cm,rmargin=1.4cm}
\usepackage{babel}
\begin{document}

\title{Guilt and Shame-Based Societies in ASOIAF}


\author{Alexander Gould}
\maketitle
\begin{enumerate}
\item Definitions

\begin{enumerate}
\item \textbf{Guilt-Based} - Centered around the individual, if you do something
wrong, you are expected to feel guilty for yourself, regardless of
what anyone else thinks. If society turns against you, you are also
able to turn against them and protest your innocence. Guilt societies
revolve around the idea of not doing bad stuff because of the potential
negative impact \textit{for you}. Think Ancient Rome, Mughal India,
any country in Europe from 1400 on, the US, etc.
\item \textbf{Shame-Based} - Centered around the collective. What you think
doesn't matter as much as what others think. You are expected to put
others before yourself, and you have no obligation to feel bad about
wrongdoing that is successfully kept secret. However, if society turns
on you, you don't have any way of turning against society. Shame societies
revolve around the idea of not doing bad stuff because of the potential
negative impact \textit{for everybody else}. Think China, Japan, Korea,
the USSR, England (to some extent) etc.
\item \textbf{Important Note} - Neither society can be objectively proven
to be better than the other. While certain advanced countries like
us embrace the ideas of guilt-based society and civic virtue as a
foundation of our culture, other developed nations with high standards
of living, like Japan, prioritize harmony, community, and a shame-based
society. Guilt-based countries, like Libya, can have low standards
of living, as can shame-based ones like the USSR/modern Russia.
\end{enumerate}
\item Westeros

\begin{enumerate}
\item \textbf{Noble Houses} - Many of the Westerosi houses have guilt cultures,
notably the Lannisters, who have such a strong sense of individuality
that they will break family ties to get what they want. Generally,
the southern kingdoms have more guilt-based societies, probably based
on their proximity to King's Landing, the center of guilt-based power
in Westeros. Houses farther away, like Tully, Greyjoy, Stark and Bolton,
have a far greater emphasis on loyalty and community.
\item \textbf{Outcasts} - The raiders of the Iron Islands and the wildings
have almost universally developed shame societies, possibly due to
the scarcity of resources and constant warfare. In addition, their
constant adversarial relationship to any individual authority from
the Seven Kingdoms might have caused this patter to strengthen, similar
to how Japan and Britain's shame societies strengthened in comparison
to perceved ``others''. Even Stannis begins to develop characteristics
of the ruler of a shame society, expecting loyalty and self policing
far more than Robert ever did, once there was a clearly defined ``other''.
\item \textbf{Transients} - Maesters, priests, and the like occupy a special
place in Westerosi society. Ostensibly outside the realm of politics,
and with no clearly defined ``other'', we see a guilt society emerge
as most maesters and priests see themselves as alone against the world.
In the case of Pycelle and Varys, and even lords like Littlefinger,
not clearly in the social structure, we see huge amounts of personal
accountability and guilt, and a complete disregard of what society
thinks. But, what's interesting to note about Littlefinger is that
as he moves up in the politics of Westeros and becomes more powerful
in the traditional sense, he starts to see himself as more accountable
to the community, and s more willing to forgive himself for secret
wrongdoings-traits that are evident of a shame society-which he didn't
do at all in the early books.
\item \textbf{Overall Consensus} - As a continent defined by power change
from invasion and steady and constant dynasties that span centuries,
we'd expect Westeros to have a similar society to nations with similar
histories, like Britain and Japan. But this isn't what happens. The
concept of the Targaryen dynasty as an ``other'', with a different
culture and a less than ideal regard for Westerosi culture, indigenous
reiligion, and human life, seems to have prompted a need for reliance
on oneself in some of the houses that hoped to topple Targaryen rule.
Success under these systems gave the serious contenders for the Iron
Throne lasting guilt-based societies. Areas left alone by the Targaryens,
such as the North, not only retained their own customs and religions,
but their shame-based societies as well.
\end{enumerate}
\item Essos

\begin{enumerate}
\item \textbf{Free Cities} - As seperate city-states born from a fight from
independence, the Free Cities are as guilt-based as you can get, emphasizing
commerce and cooperation, but not community or duty; free people are
free to come and go as they wish. The lack of community and sense
of individual needs placed first also gives us the idea that slavery
is okay. The Free Cities' foreign pilicy reflects this; they are largely
apolitical, backing whoever they think will either pay them money,
or in the case of Braavos, get Westeros to pay its debt.
\item \textbf{Slaver Cities} - The cities that line Slaver's Bay are almost
like mirror images of the Free Cities. They have economies based on
slavery and are also largely apolitical, but they have a strong sense
of identity and community. Slavery is not justified because you don't
give a crap about the people in your community, but rather because
the slaves are not in your communtity in the first place. Slaves are
almost always considered captives, and this status is passed down
from generation to generation of slaves. The dynamic here is extremley
interesting as well; the slaves hate the Masters, but they do not
reject the idea of a Meerenese community, and a good many of them
try to reintegrate into it following Daenerys' liberation of the city.
My theory is that the idea of a Ghiscari empire lives on in these
cities, and the prestige of the community is a justification for it
to place itself above all others, almost like a hyper-shame society,
and much like Old Ghis, the greatest crime is not belonging to it.
\item \textbf{Dothraki} - The Dothraki have a strange sort of hybrid between
a guilt-based society and a shame-based one. There's an extreme emphasis
on community, but it's a community held together by antagonism between
individuals. Only by achieving individually can one inspire the rivalry
that keeps the group together. The Dothraki have a society that's
so distrustful of any outsider that the community can only survive
through distrust and rivalry.
\item \textbf{Qarth and Asshai} - These are cities that are defined as outsiders
by default. As people close to the Shadow, the residents of these
cities seem to belive the social rules of the rest of the world don't
apply to them. Case in point, we've never met a truly ``normal''
person from either city. As people who identify more as outsiders
than residents of their cities, guilt-based culture reigns supreme
here, even if it's nearly unrecognizable due to the residents' blue
and orange morality.
\item \textbf{Consensus} - As a continent with a far longer history than
that of Westeros, categorized by empires rising, falling, and leaving
little pieces of identity behind, we'd expect close-knit shame societies.
But, like Westeros, the opposite happens. The sheer magnitude of so
many societies shrinks each one down to a point of almost no consequence,
meaning that the prevailing attitude on the continent is guilt-based,
as opposed to the more shame-based natural attitudes of a more politically
static Westeros.
\end{enumerate}
\end{enumerate}
\textbf{How should I compare/conclude?}
\end{document}
