%% LyX 2.0.6 created this file.  For more info, see http://www.lyx.org/.
%% Do not edit unless you really know what you are doing.
\documentclass[english]{article}
\usepackage[T1]{fontenc}
\usepackage[latin9]{inputenc}
\usepackage{babel}
\begin{document}

\title{Homework 22: Alternating Series Estimation}


\author{Alexander Gould, Section 3}

\maketitle
\textbf{Use the Root Test to see whether the following series converge.}

25. $\sum_{k=1}^{\infty}\left(\frac{1}{k}\right)^{k}$

As we get bigger, the summed term raised to its inverse approaches
0, which means we're absolutley convergent.

26. $\sum_{k=2}^{\infty}\left(\frac{k-1}{k}\right)^{k}$

Taking the limit of the summed term shows that it grows along with
$k$, showing that the sum diverges.

\textbf{Use the Comparison Test or Limit Comparison test to see if
these series converge.}

27. $\sum_{k=1}^{\infty}\frac{1}{k^{2}+4}$

The summed term is always less than $\frac{1}{k^{2}}$, which we know
decreases to 0, which means the series converges.

28. $\sum_{k=1}^{\infty}\frac{k^{2}+k-1}{k^{4}+4k^{2}-3}$

This summed term will also always be less than $\frac{1}{k^{2}}$,
which we know decreases to 0. This series also converges again.

29{*}. $\sum_{k=1}^{\infty}\frac{k^{2}-1}{k^{3}+4}$

This term will always be less than $\frac{1}{k}$, which decreases
to 0, so this series diverges due to the Harmonic Series Test.

35. $\sum_{k=1}^{\infty}\frac{1}{2k-\sqrt{k}}$

This term will always be greater than $\frac{-1}{\sqrt{k}}$, which
we know diverges.

37. $\sum_{k=1}^{\infty}\frac{\sqrt[3]{k^{2}+1}}{\sqrt{k^{3}+2}}$

This sieries diverges, as the summed term simply approaches $\frac{1}{\sqrt{2}}$

38{*}. $\sum_{k=2}^{\infty}\frac{1}{(k\ln k)^{2}}$

This summed term approaches 0, so we can say the series converges.
\end{document}
